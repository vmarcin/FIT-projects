\documentclass[11pt,a4paper,twocolumn,titlepage]{article}
	\usepackage[text={18cm,25cm}, top=2.5cm, left=1.5cm]{geometry}
	\usepackage[czech]{babel}
	\usepackage{times}
	\usepackage[utf8]{inputenc}
	\usepackage[IL2]{fontenc}
	\usepackage{amsmath,amssymb,amsthm}

	\theoremstyle{definition}
		\newtheorem{definition}{Definice}[section]
		\newtheorem{algorithm}[definition]{Algoritmus}
	\theoremstyle{plain}
		\newtheorem{sentence}{Věta}

\begin{document}
\begin{titlepage}
	\begin{center}
		{\Huge\textsc{Fakulta informačních technologií}} \\ \smallskip
		{\Huge\textsc{Vysoké učení technické v~Brně}} \\
			\vspace{\stretch{0.382}}
		{\LARGE{Typografie a~publikování\,--\,2.\,projekt}} \\ \smallskip
		{\LARGE{Sazba dokumentů a~matematických výrazů}} \\
			\vspace{\stretch{0.618}}
	\end{center}
	 {\Large {2017} \hfill {Vladimír Marcin}}
\end{titlepage}
\section*{Úvod}
V~této úloze si vyzkoušíme sazbu titulní strany, matematických vzorců, prostředí a~dalších textových struktur obvyklých 
pro technicky zaměřené texty, například rovnice (\ref{eq:one}) nebo definice \ref{def:one} na straně \pageref{def:one}.

Na titulní straně je využito sázení nadpisu podle optického středu s~využitím zlatého řezu. Tento postup byl probírán na přednášce.
\section{Matematický text}
Nejprve se podíváme na sázení matematických symbolů a~výrazů v~plynulém textu. Pro množinu $V$ označuje card$(V)$ kardinalitu $V$.
Pro množinu $V$ reprezentuje $V^*$ volný monoid generovaný množinou $V$ s~operací konkatenace.
Prvek identity ve volném monoidu $V^*$ značíme symbolem $\varepsilon$.
Nechť $V^+=V^*-\{\varepsilon\}$. Algebraicky je tedy $V^+$ volná pologrupa generovaná množinou $V$ s~operací konkatenace.
Konečnou neprázdnou množinu $V$ nazvěme \emph{abeceda}.
Pro $w \in V^*$ označuje $|w|$ délku řetězce $w$. Pro $W \subseteq V$ označuje occur$(w,W)$ počet výskytů symbolů z~$W$ v~řetězci $w$ 
a~sym$(w,i)$ určuje $i$-tý symbol řetězce $w$; například sym$(abcd,3)=c$.

Nyní zkusíme sazbu definic a~vět s~využitím balíku \texttt{amsthm}.

\begin{definition} \label{def:one}
\emph{Bezkontextová gramatika} je čtveřice $G=(V,T,P,S)$, kde $V$ je totální 
abeceda, $T \subseteq V$ je abeceda terminálů, $S \in (V-T)$ je startující symbol a~$P$ je konečná množina \emph{pravidel}
tvaru $q\colon A \rightarrow \alpha\mbox{, kde }A \in (V-T), \alpha \in V^*\mbox{ a}~q$ je návěští tohoto pravidla. 
Nechť $N=V-T$ značí abecedu neterminálů.
Pokud $q\colon A \rightarrow \alpha \in P, \gamma, \delta \in V^*,G$ provádí derivační 
krok z~${\gamma}A{\delta}\mbox{ do }\gamma\alpha\delta$ podle
 pravidla $q\colon A \rightarrow \alpha\mbox{, symbolicky píšeme }{\gamma}A{\delta} \Rightarrow \gamma\alpha\delta\ [q\colon A \rightarrow \alpha]$ nebo zjednodušeně ${\gamma}A{\delta} \Rightarrow \gamma\alpha\delta$. 
Standardním způsobem definujeme $\Rightarrow^m\mbox{, kde }m\geq0$. Dále definujeme 
tranzitivní uzávěr $\Rightarrow^+$ a~tranzitivně-reflexivní uzávěr $\Rightarrow^*$.
\medskip

Algoritmus můžeme uvádět podobně jako definice textově, nebo využít pseudokódu vysázeného ve vhodném 
prostředí (například \texttt{algorithm2e}).
\end{definition}
\begin{algorithm}
\emph{Algoritmus pro ověření bezkontextovosti gramatiky. Mějme gramatiku} $G=(N,T,P,S)$.
	\begin{enumerate}
		{\itshape\item\label{it:one}Pro každé pravidlo $p \in P$ proveď test, zda $p$ na levé straně obsahuje právě jeden symbol z~$N$}.
		{\itshape\item Pokud všechna pravidla splňují podmínku z~kroku \ref{it:one}, tak je gramatika $G$ bezkontextová.}
	\end{enumerate}
\end{algorithm}
\begin{definition}
\emph{Jazyk} definovaný gramatikou $G$ definujeme jako $L(G)=\{w \in T^*|S\Rightarrow^* w\}$.
\end{definition}
\subsection{Podsekce obsahující větu}
\begin{definition}
Nechť $L$ je libovolný jazyk. $L$ je \emph{bezkontextový jazyk}, když a~jen když $L=L(G)$, kde $G$ je libovolná bezkontextová gramatika.
\end{definition}
\begin{definition}
Množinu $\mathcal{L}_{CF} =$ \{$L|L$ {je bezkontextový jazyk}\} nazýváme \emph{třídou bezkontextových jazyků}.
\end{definition}
\begin{sentence} \label{sen:one}
\emph{Nechť $L_{abc}=\{a^nb^nc^n|n\geq0\}$ Platí, že $L_{abc} \notin \mathcal{L}_{CF}$}.
\end{sentence}
\begin{proof}
Důkaz se provede pomocí Pumping lemma pro bezkontextové jazyky, kdy ukážeme, že není možné, aby platilo, což bude 
implikovat pravdivost věty \ref{sen:one}.
\end{proof}
\section{Rovnice a~odkazy}
Složitější matematické formulace sázíme mimo plynulý text. Lze umístit několik výrazů na jeden řádek, ale 
pak je třeba tyto vhodně oddělit, například příkazem \verb|\quad|.
$$\sqrt[x^2]{y^3_0} \quad \mathbb{N} = \{0, 1, 2,\dots\} \quad x^{y^y} \neq x^{yy}  \quad z_{i_j} \not\equiv z_{ij}$$

V rovnici (\ref{eq:one}) jsou využity tři typy závorek s~různou explicitně definovanou velikostí.

\begin{eqnarray} \label{eq:one}
\bigg\{\Big[\big(a+b\big)*c\Big]^d +1 \bigg\} & = & x \\
\lim_{x \rightarrow \infty}\frac{\sin^2 x + \cos^2 x}{4} & = & y \nonumber
\end{eqnarray}

V~této větě vidíme, jak vypadá implicitní vysázení li\-mity $\lim_{n\to\infty} f(n)$ v~normálním odstavci textu. 
Podobně je to i~s~dalšími symboly jako $\sum^n_1\mbox{ či }\bigcup_{A \in \mathcal{B}}$. 
V~případě vzorce $\lim\limits_{x \rightarrow 0} \frac{\sin x}{x} = 1$ jsme si vynutili méně úspornou\linebreak 
sazbu příkazem \verb|\limits|.

\begin{eqnarray}
\int\limits^b_a f(x)\,\mathrm{d}x & = & -\int^a_b f(x)\,\mathrm{d}x \\
\Big(\sqrt[5]{x^4}\Big)' = \Big(x^{\frac{4}{5}}\Big)' & = & {\frac{4}{5}}x^{-{\frac{1}{5}}} = \frac{4}{5\sqrt[5]{x}} \\
\overline{\overline{A \vee B}} & = & \overline{\overline{A}\wedge \overline{B}}
\end{eqnarray}

\section{Matice}
Pro sázení matic se velmi často používá prostředí \texttt{array} a~závorky (\verb|\left|, \verb|\right|).

$$ \left( \begin{array}{cc}
a+b & b-a \\
\widehat{\xi + \omega} & \hat{\pi} \\
\vec{a} & \overleftrightarrow{AC} \\
0 & \beta
\end{array} \right) $$

$$\mathbf{A}=\left\|\begin{array}{cccc}
a_{11} & a_{12} & \ldots & a_{1n} \\
a_{21} & a_{22} & \ldots & a_{2n} \\
\vdots & \vdots & \ddots & \vdots \\
a_{m1} & a_{m2} & \ldots & a_{mn}
\end{array}\right\|$$

$$ \left| \begin{array}{cc}
t & u \\
v & w 
\end{array} \right| = tw - uv $$

Prostředí \texttt{array} lze úspěšně využít i~jinde.

$$\binom{n}{k} = \left\{ 
\begin{array}{l l}
\frac{n!}{k!(n-k)!} & \quad \mbox{pro } 0 \leq k \leq n \\
0 & \quad \mbox{pro $k < 0$ nebo $k > n$} \\
\end{array} \right. $$

\section{Závěrem}
V~případě, že budete potřebovat vyjádřit matematickou konstrukci nebo symbol a~nebude se Vám dařit jej nalézt v~samotném \LaTeX u{}, 
doporučuji prostudovat možnosti balíku maker \AmS-\LaTeX.
Analogická poučka platí obecně pro jakoukoli konstrukci v~\TeX u{}.
\end{document}
























