\documentclass[11pt,a4paper,titlepage]{article}
\usepackage[left=2cm,text={17cm,24cm},top=3cm]{geometry}
\usepackage[IL2]{fontenc}
\usepackage[utf8]{inputenc}
\usepackage[czech]{babel}
%\usepackage{times}

\bibliographystyle{czplain}

\begin{document}

\begin{titlepage}
  \begin{center}
    {\Huge\textsc{Vysoké učení technické v~Brně}} \\
     \medskip
    {\huge\textsc{Fakulta informačních technologií}} \\
     \vspace{\stretch{0.382}}
    {\LARGE{Typografie a~publikování\,--\,4.\,projekt}} \\
     \medskip
    {\Huge Bibliografické citace}\\
     \vspace{\stretch{0.618}}
  \end{center}
{\Large\today \hfill Vladimír Marcin}
\end{titlepage}

{\Huge\textsc{Typografia}}
\medskip

Typografia sa zaoberá problematikou grafickej úpravy tlačených dokumentov s~použitím vhodných rezov písma i~usporiadania jednotlivých
znakov a~odsekov vo vhodnej, pre čitateľa zrozumiteľnej a~esteticky akceptovateľnej forme \cite{Wiki:Typografia}. Typografia, ako určitá
forma umenia a~techniky navrhovania písiem, má dôležité miesto v~živote dizajnéra \cite{Detepe:Typografia}. Jednou z~jeho základných úloh
je výber vhodného fontu. Pod pojmom font v~typografii rozumieme kompletnú sadu znakov abecedy jednej veľkosti a~jednotného štýlu \cite{Nishikimi:Fonts}.
Typografia sa však netýka len dizajnérov či umelcov, pretože sa s~ňou stretávame dennodenne všetci. Rovnako ako štýl reči, aj kvalita
úpravy písmen na papieri môže ovplyvniť spôsob, akým ľudia reagujú na naše správy \cite{Bang:Typography}.

Názov typografia má pôvod v~dvoch gréckych slovách, ktorými sú type (v~zmysle raziť) a~graphia (písať). Z~historického hľadiska to má svoje
opodstatnenie. Prvé výtlačky totiž vznikali otlačením (kovových) liter na papier \cite{Slavicek:Typografie}. Dejiny typografie vo vlastnom
slova zmysle začínajú v~roku 1444, kedy Johannes Gutenberg vynašiel kníhtlač. Tento známy Nemec je okrem iného aj zakladateľom
polygrafického priemyslu. Ako prvý začal v~roku 1440 tlačiť z~jednotlivých liter \cite{Typografie-ikt:Historia}.

V~dnešnej dobe typograf pri svojej práci spravidla používa niektorý zo zalamovacích programov. Jedným z~nich je napríklad program 
\TeX{} \cite{Wiki:Typografia}. Ide o~typografický systém napísaný Donaldem E. Knuthem zo Stanfordskej univerzity. Je určený pre veľmi
kvalitnú sadzbu kníh s~množstvom matematiky~\cite{Cerny:Znakove_sady}. Všeobecne najznámejším formátom \TeX{}u je \LaTeX{}, ktorý bol
vyvinutý okolo roku 1985~\cite{TeX:conf}. \LaTeX{} je vysoko kvalitný open source sadzací softvér produkujúci profesionálne výtlačky
i~PDF súbory~\cite{Kottwitz:Latex}. Jadro systému \LaTeX{} tvorí prekladač jazyka \TeX{} spoločne s~nadstavbou \LaTeX{} \cite{Rybicka:Latex}.

\newpage
\bibliography{proj4}

\end{document}
